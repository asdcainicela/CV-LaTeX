% Copyright 2023 Gerald Brandi Cainicela Aquino. 
% Todos los derechos reservados. 
% Contacto: gerald.cainicela.a@gmail.com

% Copyright 2023 Gerald Brandi Cainicela Aquino. 
% Todos los derechos reservados. 
% Contacto: gerald.cainicela.a@gmail.com

\documentclass[a4paper,10pt,sans]{article} % Fornato, estilo y el tipo de documento
\usepackage[utf8]{inputenc} % Permite el uso de caracteres Unicode en el código fuente del documento.
\usepackage[T1]{fontenc} % Configura la codificación de la fuente para admitir caracteres acentuados y especiales.
\usepackage[spanish]{babel} % Adapta el documento para su uso en español, ajustando nombres y etiquetas automáticamente.
\usepackage[top=1in, left=1in, right=1in, bottom=1in]{geometry} % Ajusta los márgenes del documento a 1 pulgada en cada lado.
\usepackage{academicons} % Proporciona acceso a iconos específicos para el ámbito académico.
\usepackage{ebgaramond} % Carga la fuente EB Garamond en el documento.
\usepackage{fontawesome5} % Permite la inclusión de iconos de la popular fuente de iconos Font Awesome.
\usepackage{hyperref} % Permite la creación de enlaces y referencias cruzadas en el documento.
\usepackage{multicol} % Facilita la creación de columnas múltiples en el documento.
\usepackage{pagecolor} % Facilita el cambio de color de la página en el documento.
\usepackage{ragged2e} % Proporciona comandos adicionales para justificar el texto y controlar la alineación en el documento.
\usepackage{sectsty} % Facilita el cambio de estilos de secciones en el documento.
\usepackage{titlesec} % Ofrece herramientas para personalizar y cambiar el formato de los títulos de las secciones.
\usepackage{xcolor} % Proporciona funcionalidades adicionales para el color en el documento.
\usepackage{xstring} % Proporciona manipulación avanzada de cadenas de texto.
\usepackage{graphicx} % Permite la inclusión de gráficos en el documento.
\usepackage{changepage} % Facilita los ajustes personalizados de diseño de página.
\usepackage{ifthen} % Proporciona comandos condicionales avanzados.
\usepackage{etoolbox} % Proporciona herramientas para la manipulación de comandos y entornos LaTeX.


 
%---cambiar acorde a tus datos
\newcommand{\myName}{Gerald Brandi Cainicela Aquino}
\newcommand{\myDegree}{Ingeniería Mecatrónica}
\newcommand{\perfil}{asset/img/perfil.png}
%---
% Copyright 2023 Gerald Brandi Cainicela Aquino. 
% Todos los derechos reservados. 
% Contacto: gerald.cainicela.a@gmail.com


% Configuración de hipervínculos
\hypersetup{
	colorlinks=true,
	linkcolor=blue,
	filecolor=magenta,      
	urlcolor=blue,
	pdftitle={\myName},
	pdfpagemode=FullScreen,
}

% Definición de un color para las secciones
\definecolor{colorSection}{HTML}{2C7ECB}

% Configuración de la fuente y el estilo de las secciones utilizando el paquete sectsty
\sectionfont{\color{colorSection}\large\bfseries}

% Configuración de la apariencia de las secciones con el paquete titlesec
\titleformat{\section} 
{\color{colorSection}\Large\bfseries}
{}
{0em}
{}[\color{colorSection}\titlerule]


\newcommand{\siPerfil}{true}
\newcommand{\noPerfil}{false}

\newcommand{\miTitulo}[2]{%
	\ifthenelse{\equal{#2}{\siPerfil}}{%
		\begin{minipage}{0.76\textwidth}
			\begin{flushleft} 
				\textbf{\LARGE \MakeUppercase{\myName}}\\
				\vspace{0.2cm}
				\Large\  \textbf{\myDegree}
			\end{flushleft}
			\vspace{0.2cm}
			#1
		\end{minipage}%
		\begin{minipage}{0.4\textwidth}
			\begin{adjustwidth}{0.65cm}{}
			\includegraphics[width=0.5\linewidth]{\perfil}
			\end{adjustwidth}
		\end{minipage}%
	}{%
		\ifthenelse{\equal{#2}{\noPerfil}}{%
			\begin{flushleft} 
				\textbf{\LARGE \MakeUppercase{\myName}}\\
				\vspace{0.2cm}
				\Large\  \textbf{\myDegree}
			\end{flushleft}
			\vspace{0.2cm}
			#1
		}{}%
	}%
}


% Definición de un comando para la sección de datos personales
\newcommand{\datosPersonalesSec}[1]{%
	\section*{Datos Personales}
	\begin{tabular}{p{3cm}p{12cm}}
		#1
	\end{tabular}
}

% Definición de un comando para la sección de formación académica
\newcommand{\formacionAcademica}[5]{%
	\begin{tabular}{p{11cm}p{10cm}}
		\textbf{#1 } &  \\
		\quad\textbf{  #2 } & #4- #5 \\
		\qquad\textit{#3 } &  \\
	\end{tabular}
}

 

% Definición de la función datosPersonales que toma tres argumentos y verifica si el segundo y tercer argumento son direcciones de correo electrónico o URL antes de formatearlos
\newcommand{\datosPersonales}[3]{%
	% Imprime el texto en negrita seguido del primer argumento y un separador
	\textbf{#1:}  &
	% Verifica si el segundo argumento contiene un símbolo de correo electrónico
	\IfSubStr{#2}{@}{\href{mailto:#2}{#2}}{%
		% Verifica si el segundo argumento contiene '.com' o '.pe'
		\IfSubStr{#2}{.com}{\href{#2}{\url{#2}}}{%
			\IfSubStr{#2}{.pe}{\href{#2}{\url{#2}}}{#2}%
		}
	} \\
	% Verifica si el tercer argumento está vacío, si no, verifica si es un correo electrónico o una URL y lo formatea como un enlace
	\ifx\empty#3\else & 
	\IfSubStr{#3}{@}{\href{mailto:#3}{#3}}{%
		\IfSubStr{#3}{.com}{\href{#3}{\url{#3}}}{%
			\IfSubStr{#3}{.pe}{\href{#3}{\url{#3}}}{#3}%
		}
	} \\ 
	\fi
}

\newcommand{\noObjetivo}{

}

\newlength{\mySpaceTask}
\setlength{\mySpaceTask}{0.3889em} % denota 7/18, es decir \:\,  0.3889em 4+3/18
\newcommand{\task}[1]{%
	%\hspace{\mySpaceTask} 
	
	\begin{minipage}{0.2cm} 
		\:
	\end{minipage}
	\begin{minipage}{\mySpaceTask} 
			\textbullet
	\end{minipage}
	\begin{minipage}{\dimexpr\textwidth-\mySpaceTask-2cm\relax}
	 #1
	\end{minipage}

}

\newcommand{\taskV}[1]{%
	%\hspace{\mySpaceTask} 
	
	\begin{minipage}{0.2cm} 
		\:
	\end{minipage}
	\begin{minipage}{\mySpaceTask} 
		\,
	\end{minipage}
	\begin{minipage}{\dimexpr\textwidth-\mySpaceTask-2cm\relax}
		#1
	\end{minipage}
	
}


	%\:=4/18
	%\, =3/18
\newcommand{\experienciaProyecto}[6]{%
	\begin{tabular}{p{12.4cm}p{3cm}}
		\textbf{#1} &\\
		#2 & #3 - \\
		\ifx#5\noObjetivo
		\else
		\textbf{Objetivo: } #5 & #4\\
		\fi
	\end{tabular} 
	
	\medskip
	#6
	\vspace{0.5cm}
}
\newcommand{\experienciaProyectoV}[6]{%
	\begin{tabular}{p{9.6cm}p{4.4cm}}
		\textbf{#1} &\\
		\textbf{#2} & \hfill #3 - #4
	\end{tabular}
	
	\ifx#5\noObjetivo
	\else
	\begin{tabular}{p{15.4cm}}
		\textbf{Objetivo: } #5
	\end{tabular} 
	\fi
	
	\smallskip
	#6
	\vspace{0.5cm}
}
 
 
\newcommand{\experienciaLaboral}[5]{%
	\begin{tabular}{p{9.6cm}p{4.4cm}}
		\textbf{#1} &\\
		\textbf{#2} & \hfill #3 - #4
	\end{tabular}
	
	\smallskip
	#5
	\vspace{0.5cm}
}



\newcommand{\voluntariadoAcademico}[5]{%
	\begin{tabular}{p{9.6cm}p{4.4cm}}
		\textbf{#1} &\\
		\textbf{#2} & \hfill #3 - #4
	\end{tabular}
	
	\smallskip
	\,\: #5
	\vspace{0.2cm}
	
}



\newcommand{\competenciasDigitales}[1]{%
	\begin{tabular}{p{5cm}p{10cm}}
		#1
	\end{tabular}
}
\newcommand{\habilidadesPersonales}[1]{%
	\begin{tabular}{p{5cm}p{10cm}}
		#1
	\end{tabular}
}
\newcommand{\idiomas}[1]{%
	\begin{tabular}{p{4cm}p{12cm}}
		#1
	\end{tabular}
}
\newcommand{\taskC}[2]{% 
	\textbullet	\ #1: & #2\\
}
\begin{document}
	 
	%\justify %justificar 
	%\pagecolor{gray!5} % Cambia el color de fondo aquí

	\miTitulo{
		Soy una persona apasionada por la innovación y la tecnología, con sólida formación académica y habilidades en resolución de problemas, trabajo en equipo y liderazgo. Soy proactivo y adaptable a nuevos desafíos. Busco contribuir al desarrollo tecnológico y crecer profesionalmente en una empresa líder en ingeniería.
		}{true} % 	true: si tiene alguna foto
				%	false: no tienes ninguna foto

	\datosPersonalesSec{
		\datosPersonales{Correo(s)}{gerald.cainicela.a@gmail.com}{gerald.cainicela.a@uni.pe}
		\datosPersonales{N° Celular(es)}{930943503}{939402816}
		\datosPersonales{LinkedIn}{www.linkedin.com/in/gerald-cainicela}{}
		\datosPersonales{GitHub}{https://github.com/asdcainicela}{}
	}

	\section*{Formación Académica}
	
		\formacionAcademica
		{Universidad Nacional de Ingeniería}
		{Egresado en \myDegree}
		{Quinto Superior}
		{03/2019}{7/2023}

	\section*{Experiencia Laboral}
		\experienciaLaboral
		{Mining Mechatronick}
		{Ingeniero de Control y Automatización}
		{01/08/2023}{Actualidad}
		{
			\task{Encargado del desarrollo de proyectos de automatización para maquinaria utilizada en minería subterránea, brindando soluciones innovadoras y eficientes para optimizar los procesos en este entorno exigente. }
			\task{Especialista en redes industriales CANbus, incluyendo los protocolos CANopen y J1939, lo que permite implementar soluciones de comunicación confiables y robustas en nuestros sistemas de automatización. }
			\task{Experto en programación de HMI's, controladores y gateways IoT mediante el uso de CoDeSys 3.5, proporcionando interfaces amigables y funcionales para nuestros clientes y garantizando la integración de los dispositivos en nuestros proyectos. }
			\task{Proveer un soporte técnico especializado para los dispositivos de automatización ofrecidos por la empresa, asegurando un acompañamiento integral a nuestros clientes durante todas las etapas de implementación y mantenimiento, y garantizando su satisfacción con nuestros productos y servicios.}
		}

		\experienciaLaboral{Multi Print}
		{Practicante en Mantenimiento de Equipos Electrónicos}
		{01/03/2023}{30/06/2023}
		{
			\task{Realización de mantenimiento preventivo y correctivo en máquinas impresoras VP's-Canon.}
			\task{Ajuste y calibración de las máquinas para garantizar un rendimiento óptimo.}
			\task{Inspección y diagnóstico de posibles problemas y fallas en las máquinas.}
			\task{ Reparación y reemplazo de componentes defectuosos.}
			\task{Asesoramiento a los clientes sobre el uso adecuado y el mantenimiento básico de las máquinas VP's-Canon.}
			\task{ Documentación de los trabajos realizados y elaboración de informes.}
		}

	\newpage

	\section*{Experiencia en Proyectos}

		\experienciaProyectoV
		{Chasqui II, UNI, CTIC, Smart Machine}
		{Miembro del Área de Payload}
		{30/06/2023}{Actualidad}
		{Desarrollo de un CubeSat para el concurso de Asia Pacific Space Cooperation Organisation (APSCO)} % si no tiene ningún objetivo, debe poner \noObjetivo
		{
			\task{Participación en el diseño y construcción de la tarjeta electrónica CubeSat}
			\task{Contribución a la programación y configuración de la carga útil (payload) del CubeSat. }
			\task{Colaboración en la integración de sistemas y pruebas de funcionalidad}
			\task{Preparación del CubeSat para la participación en el concurso APSCO.}
			\task{Trabajo en equipo para garantizar el éxito del proyecto.}
		}

		\experienciaProyectoV{Wayta Wayra, UNI, UNIFIM, CEDIME}
		{Director de Proyecto}
		{04/07/2023}{Actualidad}
		{Diseño y Desarrollo de Aerogeneradores Savonius para la Recarga de Bicicletas y Vehículos Eléctricos, así como para la Mejora de la Iluminación Pública en las Avenidas Principales de Lima. }
		{
			\task{Participación en el diseño y construcción de la tarjeta electrónica CubeSat}
			\task{Utilización de herramientas como CAE (Computer-Aided Engineering) y CAD (Computer-Aided Design) con SolidWorks y Ansys. }
			\task{Coordinación con los equipos multidisciplinarios para asegurar la optimización del rendimiento y la eficiencia energética. }
			\task{Colaboración y comunicación efectiva con los equipos internos y stakeholders para cumplir con los objetivos planteados.}
		} 
		
		\experienciaProyectoV{Qhapaq Ñan Project, CTIC, Smart Machine}
		{Miembro del Área de PayLoad}
		{25/07/2023}{Actualidad}{Desarrollo de tarjeta electrónica CubeSat para el proyecto BIRDS-X APRS Payload Competition}
		{
			\task{ Responsable del magnetómetro RM3100 y programación en C y Python para Raspberry Pi Pico. }
			\task{Gestión de la memoria flash FM MT25QL01GBBB8ES en la tarjeta electrónica. }
			\task{Participación en el diseño y desarrollo del PCB y del TNC ATMEGA644P.}
		}
		
		
		\experienciaProyectoV{Electromovilidad UNI, IEEE VTS UNI, UNI}
		{Subdirector Técnico de Electromovilidad}
		{29/07/2023}{Actualidad}
		{Diseño y Desarrollo de Aerogeneradores Savonius para la Recarga de Bicicletas y Vehículos Eléctricos, así como para la Mejora de la Iluminación Pública en las Avenidas Principales de Lima. }
		{
			\task{Liderazgo en la creación y publicación de artículos técnicos sobre electromovilidad e hidrógeno verde.}
			\task{Colaboración activa con expertos en el campo para garantizar la precisión y relevancia de los contenidos. }
			\task{Participación en conferencias y simposios para presentar investigaciones y avances en electromovilidad. }
			\task{Coordinación de esfuerzos con el capítulo local de IEEE VTS UNI para difundir conocimiento en movilidad eléctrica y sostenible. }
		}
		 
		\experienciaProyectoV{Acreditación UNI 2023-1, ABET, UNI, FIM}
		{Proyectista ABET en Análisis Y Control de Robots}
		{4/05/2023}{16/07/2023 }
		{\noObjetivo}
		{
			\task{ Participación en la Feria de Proyectos ABET.}
			\task{Proyecto titulado ``Robot de 2 grados de libertad con control PID orientado al riego tecnificado".} 
		} 
		
		\experienciaProyectoV{Acreditación UNI 2022-1, ABET, UNI, FIM}
		{Proyectista ABET en Procesadores Digitales de Señales}
		{4/10/2022}{16/12/2022}
		{\noObjetivo}
		{
			\task{  Participación en la Feria de Proyectos ABET.}
			\task{  Proyecto titulado ``Diseño de un prototipo de arranque de bomba para proceso de Líquidos de Gas Natural (NGL) con el microprocesador TMS320F28335".}
		} 
		
		\experienciaProyectoV{Acreditación UNI 2021-2, ABET, UNI, FIM}
		{Proyectista ABET en Sistemas Embebidos}
		{10/09/2021}{15/11/2021}
		{\noObjetivo}
		{
			\task{  Participación en la Feria de Proyectos ABET.}
			\task{  Proyecto titulado ``Control de Temperatura para una incubadora de huevos mediante IoT con el PIC16F877A". }
		} 
	
	\section*{Voluntariado Académico}
		\voluntariadoAcademico{UNI - CEFIM}{Asesor Académico}{11/2022}{01/2023}
		{Elaboración de material de estudio y dictado de clases de Cálculo Integral para estudiantes de la UNI-FIM}
		\voluntariadoAcademico{UNI - TEFIM}{Asesor Académico}{ 06/2022}{ 07/2022}
		{Elaboración de material de dictado de clases de Cálculo Diferencial e Integral para estudiantes de la UNI-FIM.}
		\voluntariadoAcademico{UNI - CEFIM}{Docente Académico }{11/2021}{01/2022}
		{Elaboración de material de estudio y dictado de clases de LaTeX para estudiantes de la UNI o externos.}
		\voluntariadoAcademico{UNI - RA OPT-FIM }{Tutor}{03/2023}{07/2023}
		{Elaboración de material de estudio y dictado de clases de Cálculo Integral para estudiantes en riesgo académico.}

	
	\section*{Competencias Digitales}
		\competenciasDigitales{
			\taskC{ Herramientas de Microsoft Office}{Excel, Power BI, Power Query, Word, PowerPoint, Microsoft Teams}
			\taskC{Simulación y Diseño de Sistemas}{Simulink, Fluid Sim, Proteus} 
			\taskC{Análisis y Modelado CAE}{ANSYS, Patran y Nastran} 
			\taskC{Microcontroladores}{PLC, DSP, Arduino (Uno y Leonardo),  Raspberry Pi Pico W, ESP32, ESP8266, PIC16F877A}
			\taskC{Lenguajes de Programación}{Matlab, C, C++, Python, Java }
			\taskC{Entorno de Desarrollo de PLC}{TIA Portal, CODESYS 3.5}
			\taskC{Diseño 3D y Modelado}{AutoCAD, SolidWorks ,Autodesk Inventor, Autodesk Maya}
			\taskC{Software de Mecanizado}{Mastercam X5, SSCNC }
			\taskC{Herramientas Configuración CAN}{EPEC Multi Tool, PCAN-View, CanMoon.}
			\taskC{Protocolos CAN}{CANbus, CANopen, J1939 }
			\taskC{Composición de Documentos}{\LaTeX}
		}
	

	
	\section*{Habilidades Personales}
		\habilidadesPersonales{
			\taskC{Adaptabilidad:}{ Flexibilidad para enfrentar desafíos y ajustarse a cambios.}
			\taskC{Comunicación efectiva}{Capacidad para transmitir ideas y conceptos de manera clara y concisa.} 
			\taskC{Innovación y Creatividad:}{ Capacidad para generar ideas originales y soluciones creativas.} 
			\taskC{Liderazgo y Capacidad de Gestión:}{Experiencia en guiar equipos hacia el logro de objetivos.} 
			\taskC{ Relaciones Interpersonales:}{ Competencia en establecer y mantener relaciones efectivas.} 
			\taskC{Organización:}{ Habilidad para planificar y coordinar tareas de manera efectiva.}
			\taskC{Pensamiento Analítico:}{ Capacidad para descomponer problemas y tomar decisiones informadas. }
			\taskC{Resiliencia:}{antener calma y productividad bajo presión. }
			\taskC{Resolución de Problemas:}{Habilidad para analizar situaciones complejas y generar soluciones efectivas. }
			\taskC{Toma de Decisiones: }{Capacidad para evaluar opciones y tomar decisiones informadas. }
			\taskC{Trabajo Bajo Presión: }{Habilidad para mantener el enfoque y la eficiencia en situaciones desafiantes. }
			\taskC{Trabajo en Equipo: }{Colaborar en equipos multidisciplinarios para lograr objetivos comunes. }
		}
 
	\section*{Idiomas}
		\idiomas{
			\taskC{Español}{ Nativo }
			\taskC{Inglés}{ Intermedio }
			\taskC{Portugués}{ Intermedio}
			\taskC{Chino}{Básico} 
		}
 
\end{document}